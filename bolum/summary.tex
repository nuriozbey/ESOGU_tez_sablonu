\chapter{SUMMARY}
\lipsum[7-9]


% Yüz ifadesi tanıma ve analiz etme son yıllarda popüler bir konu haline geldi. Önceden insan-makine etkileşimini, güvenlik, psikolojik analiz ve eğlence amaçlı kullanılırken, son zamanlarda web sitesi ve pazarlama uygulamaları ile sosyal medyada eğlence uygulamaları araştırmacıları bu konu üzerine çalışmaya yoğunlaştırdı. Literatürdeki çalışmalar artmasına rağmen bu konudaki tanıma hassasiyeti ve zorluğu halen devam etmekte. Bu konudaki çalışmaları ve tanımaları farklılaştıran en büyük etki insanın doğal yapısı gereği ortaya çıkan farklı ırklardaki farklı yüz tiplerinin oluşması olmuştur. Tek bir tipteki ifade tanıma başarısı çok iyi olmasına rağmen farklı tiplerde ortaya çıkan tanıma zorluğu halen bir yarış olarak devam etmektedir.

% Gerçekleştirilen bu tez çalışmasında literatürdeki yapılan çalışmalara ek yeni bir öznitelik çıkarımı ve veri seti üzerindeki dengeleme ile daha hassas ve daha başarılı bir yöntem önerilmiştir. Bu yöntemler kıyaslanırken video olarak geçmişe dayalı bir sınıflandırma yerine tek bir resim üzerinden sınıflandırma gerçekleştirilmeye çalışılmıştır. Çalışmanın ilk aşaması olarak ayırt edici özellik olarak öznitelik türetme ile başlamıştır. Bu aşamada üç farklı yöntem olarak, geometrik öznitelikler, \acrfull{hog} ve \acrfull{lbp} kullanılmıştır. Geometrik öznitelik türetme aşamasında, yeni yaklaşımlar önerilmiş ve bu öznitelikler türetilirken (Özbey ve Gülmezoğlu, 2021) çalışmasındaki yaklaşımlardan yararlanılmıştır. Deneysel çalışmalarda genişletilmiş \acrfull{ckplus} veri kümesi kullanılmış ve yüz ifadeleri 10 katlı çapraz doğrulama yöntemi kullanılarak 4 farklı sınıflandırıcı; \acrfull{svm}, \acrfull{rf},\acrfull{logreg} ve \acrfull{cva} ile sınıflandırılmıştır. Sınıflandırma başarısını arttırmak için \acrfull{sffs} , \acrfull{sbfs} ve \acrfull{pca} ile öznitelik azaltma işlemi yapılmıştır. 

\noindent
\textbf{\textit{Keywords:}}
One, two, three, four, five